%!TEX root = ../article.tex

% \listoftodos

%1
\levelA{\label{chap:introduction}Introduction}
\input{content/1-introduction.tex}

    %\levelB{Motivation}
    In file fragment classification, the objective is to identify to which file type a piece of data belongs. It is a relevant task in file recovery, and researches in this subject did not provided a solution that could substitute the software used in practice, still manually coded. Since some of those tools support hundreds of file types, a machine learning solution that could learn the patterns from examples instead of relying on specifications should be beneficial.



% %from pep 1, paragrafo 1
% In a forensic context, file recovery is a frequent task that can be motivated by several situations, like physical media malfunction, intentional attempt to hide data, and the need to access deleted or older versions of files. When the filesystem no longer provides the physical location of a file on the media, data carving is often the only procedure capable of retrieving this content.

% %from pep 1, paragrafo 2
% Data carving is a forensic process that attempts to recover files without previous information of where the file starts or ends \cite{garfinkel_carving_2007}.
% To accomplish this, a program has to analyze a source of raw data, searching for patterns indicating a known file type and making attempts to locate and reconstruct each of its constituent parts.

% %from pep 1, paragrafo 2b
% That process commonly disregards the filesystem \cite{veenman_statistical_2007}, being able to recover deleted files from unallocated areas, but faces the problem of fragmentation \cite{veenman_statistical_2007}  \cite{pal_evolution_2009}: in many cases, files are not written sequentially on disk and deleted files may have missing parts.

This type of file recovery, called file carving or data carving, is frequently used in forensic environments, but it is also used in other areas, such as reverse engineering, network traffic analysis, and data mining.

% This observation is related to the fact that many types of data sources contain embedded files. Therefore, they may be used as input to a data carving process. This includes network traffic, memory dumps, hard drive images, and files containing other files.


    % \input{content/1.1b-example.tex}
    % \input{content/1.1c-datacarvingclasses.tex}

    

% Misclassification analysis should play a prominent role in published classification research. In the past, researchers have focused on building classifiers without discussing classification errors, which significantly degrades the contribution to the body of knowledge. Such discussion helps us explain classifier performance 

% The results show
% the classifier more accurately classifies low entropy data, similar to past research findings.


% Scalability could also be improved via a hierarchical classi-
% fication approach, since it reduces the number of classes. In a
% hierarchical classification approach, a file fragment is classified
% into one of several large categories and then further classified
% into a specific type within the category, which may not even be
% needed in all cases.

    %\levelB{Research questions}

    \subsection{Research question}
% This work compares the use of Multilayer Perceptrons (MLP), Convolutional Neural Networks (CNN) and Long Short-Term Memory (LSTM), and combinations of those types of networks, to perform file fragment classification, which is the first step of the data carving process, identification, answering the following initial questions:

The study of Beebee et al. \cite{beebe_sceadan:_2013} is one of the best works on file fragment classification: it emphasized the problem of multiple data types in a file type, achieved high accuracy using a large amount of file types, made their source code available, suggested hierarchical classification as a way to improve scalability, suggested that misclassification analysis should play a prominent role in classification research, endorsed that high entropy file types tend to have a lower classification accuracy, and briefly mentions that the number of classes should be taken in consideration when comparing accuracy of different studies.

This research addresses specifically the last item in the list, the influence of number of classes in accuracy, but the two previous items, misclassification analysis and high entropy, are also part of the discussion. Although is intuitive that a higher number of classes should have an influence on the results of a file fragment classifier, the issue, considering the carving problem perspective, wasn't formally studied yet. Thus, the following initial question is proposed:

%from pep 4.1
\begin{enumerate}[itemindent=\parindent,label=\textbf{Q\arabic*.}]

% Then, the influence of the number of classes on the accuracy of the resulting models is briefly explored.
    \item How does the accuracy of neural network models change relative to the number of classes?

\end{enumerate}

% In this work, only neural networks are implemented, but other machine learning approaches exist, like Support Vector Machines (SVM) and k-Nearest Neighbors (kNN). This choice of restriction was motivated by the success that deep neural networks have shown in other fields, like image classification and speech recognition.
% The flexibility of neural networks to combine different types of layers is also important, as it is a core characteristic being explored in this research.

% Two sets of experiments were conducted. In the first set, the initial 512 bytes of a file is used as input to the tested neural network, whose task is to predict the file type. The second set is similar, but the 512 bytes fragment is extracted from a random position of the file, which is a more difficult task as it cannot depend on header patterns.

    %\levelB{Overview}

    % \subsection{Outline}
% providing an overview of the dissertation or report structure

The remainder of this document is organized as follows.
% \todo{check later}
    Section 2 analyses current researches on file fragment classification using the Govdocs1 dataset. 
    Section 3 describes the proposed method.
    Section 4 presents the results.
    Section 5 discusses the results, including suggestions for future work.
    Section 6 presents the conclusion.



%3
\levelA{Related work}
\input{content/2-relatedwork.tex}
    \label{sec:relatedwork}
    % \levelB{From PEP}
    \input{content/2.1b-resultados-outros-trabalhos.tex}

    % \levelB{Current data carving tools}
    % \input{content/2.2-datacarvingtools.tex}
    % \levelB{Data carving challenges}
    % \input{content/2.3-challenges.tex}
    % \levelB{Neural networks researches in data carving}
    % \input{content/2.4-nn.tex}

%2
% \levelA{\label{chap:background}Some artificial neural networks}
% \input{content/3-background.tex}
%     \levelB{\label{sec:feedforward}Multilayer Perceptron}
%     \input{content/3.2-feedforward.tex}
    
%     \levelB{\label{sec:conv}Convolutional neural network}
%     \input{content/3.3-conv.tex}
    
%     % \levelB{\label{sec:rnn}Recurrent Neural Network}
%     % \input{content/3.4-rnn.tex}
    
%     \levelB{\label{sec:lstm}Long Short-Term Memory}
%     \input{content/3.5-lstm.tex}

%4
\levelA{\label{chap:experiments}Proposed method}

\input{content/4-0.tex}

\levelB{Method}
\input{content/4-method.tex}

\levelB{Results}
\label{sec:numberofclasses}
\input{content/4-results.tex}

%5
\levelA{\label{chap:discussion}Discussion}

%fig 1
\levelB{Accuracy vs. number of classes}

In figure \ref{fig:nclasses}, a decreasing trend was observed. An increase in the number of classes appears to be  correlated with a decrease in accuracy. Another relevant aspect of the graph is that the range of the results seems to be smaller when more classes are used.  

This pattern is understandable: as the number of classes grows, the harder the classification problem is, leading to a decrease in accuracy. Meanwhile, the individual contributions of each class to the overall result diminishes, leading to an increase in precision.

This behavior is an important aspect to consider during the evaluation of file fragments studies. This observation is in agreement with Beebe et al. \cite{beebe_sceadan:_2013} assumption, that studies that select fewer classes tend to yield higher results. 

Still, with 42\% of samples being misclassified when the number of classes is 28, the question of what are the error sources and how they can be addressed requires attention.

The number of possible combinations of file types to compose the datasets depends on the number of classes being considered. For 28 classes there is only one possible combination, while for two classes there are 378. For intermediary values, the numbers are much higher, which is the number of possible combinations disregarding the order of the elements: $ \frac{28!}{(28-n)!n!}$. For 14 file types there are 40116600 combinations. For this reason, the significance of the 5 samples diminishes for intermediary values.

\levelB{Accuracy of pairs of classes}

The accuracy of models trained with pairs of classes, shown in figure \ref{fig:dual}, suggests a reverse correlation between entropy and accuracy.  Generally, file types with higher entropy tend to have lower minima, with the GIF file type being a notable exception. Most of these files use some form of compression, like image files for example.

It was demonstrated that the accuracy of a new model may be manipulated by the selection of file types that will compose the dataset. The lines ``hard file types first'' and ``easy file types first'' of figure \ref{fig:nclasses} were created using the order shown on figure \ref{fig:dual}, resulting in lines that seem to be close to the minimum and maximum of the possible accuracy values. 

\levelB{PCA}

The usage of PCA on the 28x28 distance matrix produced a 2D projection where the file types that use compression or contains images are grouped near each other, as shown in figures \ref{fig:pca} and \ref{fig:pca2}.


\levelA{\label{chap:conclusion}Conclusion}
It was observed that an increase in the number of extensions selected to compose the training had the tendency to decrease accuracy and increase precision. But the number of classes alone is not as important as the type of extension selected: some file types when included in the experiment have a much higher negative impact than others. This observation was demonstrated in the ``hard file types first'' and ``easy file types first'' of figure \ref{fig:nclasses}, where the file types selected to compose  where intentionally chosen, once to degrade results and once to improve them.

File types that contains images or that use compression were identified as those that have the higher negative effect on results, which suggests that their entropy may contribute to the error.

\levelB{Limitations, threats to validity and future work}

The number of samples taken was small when compared with the number of all possible file types combinations. This imposes a limit on the conclusions that can be reached and this limitation is hard to overcome.

The group that emerged as file types that most degrade results are files that use compression or contain images. While they are known for their high entropy, no measure of entropy was used to reinforce this claim. A study is in progress to measure the impact of entropy on file fragment classification errors.

