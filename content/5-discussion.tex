%fig 1
\levelB{N classes}

\ref{fig:nclasses}

An increase in the number of classes appears to be  correlated with a decrease in accuracy. Another relevant aspect of the graph is that the range of the results seems to be smaller when more classes are used.  

This pattern is understandable: as the number of classes grows, the harder the classification problem is, leading to a decrease in accuracy, while the individual contributions of each class to the overall result diminishes, leading to an increase in precision.

This behavior is an important aspect to consider during the evaluation of file fragments studies. As Beebe et al. \cite{beebe_sceadan:_2013} have mentioned, studies that select fewer classes tend to yield higher results. 

Still, with 46\% \todo{check} of samples being misclassified when the number of classes is 28, the question of what are the error sources and how they can be addressed requires attention.

\levelB{Pairs of classes}

\ref{fig:dual}

Generally, file types with higher entropy tend to have lower minima, with the GIF \todo{check} file type being a notable exception. Most of these files use some form of compression, like image files for example. 

\levelB{PCA}

\ref{fig:pca} and \ref{fig:pca2}

Again some file types with high entropy emerge as a promising group, \todo{check} ``dwf'',
``jpg'',
``pps'',
``ppt'',
``gz'',
``png'',
``pptx'',
``swf'',
``kmz'',
and ``pdf''.
